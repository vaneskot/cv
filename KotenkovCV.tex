\documentclass[11pt,a4paper]{moderncv}
\usepackage[T2A]{fontenc}
\usepackage[english, russian] {babel}
\usepackage[utf8]{inputenc}

\moderncvstyle{classic}
\moderncvcolor{blue}
\renewcommand{\rmdefault}{cmr}

% adjust the page margins
\usepackage[scale=0.75]{geometry}

% personal data
\firstname{Иван}
\familyname{Котенков}
\mobile{+7~(905)~790~2816}
\email{ivan.kotenkov@gmail.com}

\begin{document}
\makecvtitle

\section{Образование}
\cventry{2005 -- 2012}{Специалист}{Московский Государственный Университет им. Ломоносова}
  {\newline\textit{Факультет}: механико-математический}
  {\newline\textit{Специальность}: прикладная математика}{}

\section{Опыт работы}
\subsection{Работа в офисе}
\cventry{Сен 2013 -- н.в.}{Старший разработчик}{\textbf{Яндекс}}{Москва}{}{Разработка рендеринг-движка Яндекс.Браузера.}
\cventry{Июл 2013 -- Авг 2014}{Разработчик}{\textbf{Яндекс}}{Москва}{}{Разработка рендеринг-движка Яндекс.Браузера.}
\cventry{Авг 2010 -- Июн 2013}{Инженер 1 кат.}{\textbf{ОАО НПО ЛЭМЗ}}{Москва}{}{Разработка программного обеспечения для управления воздушным движением.\newline{}%
Обязанности:
\begin{itemize}%
\item Разработка системы поиска пересечений 3d объектов.
\item Разработка системы визуализации 3d объектов.
\item Реализация сопряжения между серверами по протоколу Asterix.
\end{itemize}}
\cventry{Июль 2008 -- Фев 2009}{Разработчик C++}{\textbf{Рок Флоу Динамикс}}{Москва}{}{Модификация и поддержка сервера лицензий. Работа с USB-ключом.}
\subsection{Удаленная работа}
\cventry{Июль 2012 -- н.в.}{Разработчик C++}{\textbf{SimplyCeph}}{}{}{Обработка 3d моделей (stl).}

\section{Навыки}
\cvlistitem{\emph{Основной язык программирования}: C++}
\cvlistitem{\textit{Остальные языки программирования}: C, Python}
\cvlistitem{\textit{Библиотеки}: STL, Qt}
\cvlistitem{\textit{Остальное}: linux, mac os, vim, bash, svn, git, tex}

\section{Языки}
\cvitemwithcomment{Русский}{Родной}{}
\cvitemwithcomment{Английский}{Продвинутый}{}
\cvitemwithcomment{Чешский}{Продвинутый}{}

\renewcommand{\listitemsymbol}{-~}

\end{document}
