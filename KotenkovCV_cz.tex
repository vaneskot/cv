\documentclass[11pt,a4paper]{moderncv}
\usepackage[T2A]{fontenc}
\usepackage[czech] {babel}
\usepackage[utf8]{inputenc}

\moderncvstyle{classic}
\moderncvcolor{blue}
\renewcommand{\rmdefault}{cmr}
\usepackage[scale=0.75]{geometry}

% personal data
\firstname{Ivan}
\familyname{Kotenkov}
\mobile{+420777001876}
\email{ivan.kotenkov@gmail.com}

\begin{document}
\makecvtitle

\section{Vzdělání}
\cventry{2005 -- 2012}{Specialista}{Moskévská statní univerzita}
  {\newline\textit{Fakulta}: Mechaniko-matematická}
  {\newline\textit{Studijní program}: Matematika}
  {}

\section{Ročniková práce}
\cvline{Téma}{\emph{Fault-tolerant process manager for MPI jobs in display classes}}
\cvline{Abstrakt}{\small Během práce jsem se vyznal v zdrojovém kódu MPICH a udělal v něm některé změny, aby byl odolnější proti selhání.}

\section{Diplomová práce}
\cvline{Téma}{\emph{Functions interpolation based on monotonicity property}}
\cvline{Abstrakt}{\small
V práci je rozebrán problém, vznikající při připravení 3D modelu ropného naleziště z geologických a geofyzikálních dat.
Během ročnikové práce jsem prostřednictvím minimalizace některého funkcionálu získal vzorce pro interpolace hranic vrstvy, aby se neprotínaly.
V diplomové práci jsem použil tyto výsledky a zjistil jsem řešení v případě některých typů poruch vrstv.
Vývojal jsem také program v C++ pro řešení a vizualizace tohoto problému, v kterém jsem použil několik metod řešení soustav lineárních rovnic.
}

\section{Pracovní zkušenosti}
\cventry{\small{Zaří 2014 -- Současnost}}{Senior Programátor}{\textbf{Yandex}}{Moskva}{}{Vývoj jádra renderování webového prohlížče Yandex.Browser.}
\cventry{\small{Červenec 2013 -- Srpen 2014}}{Programátor}{\textbf{Yandex}}{Moskva}{}{Vývoj jádra renderování webového prohlížče Yandex.Browser.
\begin{itemize}%
\item Účast v vývoju a architektuře skoro všech užitných vlastností, které se tykají jádra renderování.
\item Upstream patchů v Chromium.
\item Recenze kódu.
\item Vedení přijímacích pohovorů.
\end{itemize}
}
\cventry{\small{Srpen 2010 -- Červen 2013}}{Programátor}{\textbf{Lianozovská Elektromechanická Tovarna}}{Moskva}{}{Vývoj systémů řízení letového provozu.
\begin{itemize}%
\item Vývoj knihovny pro hledaní průniku trojrozměrných objektů.
\item Vývoj systému vizualizace trojrozměrných objektů.
\item Vývoj spojení mezi servery s využitím protokolu ASTERIX.
\end{itemize}
}

\cventry{\small{Červenec 2012 -- Únor 2013}}{Programátor}{\textbf{SimplyCeph}}{Vzdálená práce}{}{Vývoj programu pro modifikaci STL(STereoLithography) modelů: automatické odstranění zbytečních částí, hledaní dír.}
\cventry{\small{Červenec 2008 -- Únor 2009}}{Programátor}{\textbf{Rock Flow Dynamics}}{Moskva}{}{Oprava programátorských chyb a zlepšení serveru klíčů.}

\section{Technologie}
\cvlistitem{\textit{Programovací jazyky}: C++, C, Python}
\cvlistitem{\textit{Knihovny}: STL, QT}
\cvlistitem{\textit{Operační systémy}: linux, mac os}
\cvlistitem{\textit{Kontrola verzí}: git, svn}
\cvlistitem{\textit{Ostatní}: gdb, lldb, vim, bash, tex}

\section{Jazyky}
\cvitemwithcomment{Ruština}{Mateřský jazyk}{}
\cvitemwithcomment{Angličtina}{Dobře}{}
\cvitemwithcomment{Čeština}{B2}{}

\renewcommand{\listitemsymbol}{-~}

\end{document}

